The main data file for this specification is a Comma Separated Value file that
has the following requirements.  The last data column(s) may reference one or
more external files. These referenced files can be of any file type or file
extension.  Note that \textit{new data can be added to or deleted from the
\texttt{\small csv} file or files can be added to or removed from the database
i.e., the \texttt{\small database\_name.csv} directory, as long as the 
following requirements are met.} The specification for the Spec D v1.1
\texttt{\small csv} follows (differing from v1.0 in points 5 and 9).

\begin{enumerate}
\item The file is UTF-8 encoded.
\item The file follows the specification rfc4180 \cite{rfc4180}, governing how
fields are defined.
\item The first line of the file (header) is \textbf{required}.  
\item Header values are unique non-empty \textbf{strings} that are the labels 
for the columns.
\item The last header values may be or start with \texttt{\small FILE}. No 
other non-\texttt{\small FILE} columns can occur after the first
\texttt{\small FILE} columns (\texttt{\small FILE} columns are always last). 
\item All other lines in the file are data rows.
\item The first data line (second line in the file) is \textbf{required}.
\item Each data line is populated with an equal number of values
(columns) as the header (i.e., the same number of separating commas per row).
\item In the data lines:
\begin{enumerate}
\item All data values must be \textbf{floats} (floating-point values), 
\textbf{integers}, \textbf{strings}, or \textbf{empty} (missing) values.
\item Missing (\textbf{empty}) values may not occur in the first data row 
(second line).
\item All data values in \texttt{\small FILE} columns are \textbf{required}
to be \textbf{strings} or \textbf{empty} values.
\item All other columns are \textbf{required} to have the same value type for
all data rows, either \textbf{float}, \textbf{integer}, or \textbf{string}.
\item A column may not have an \textbf{empty} value type, but \textbf{empty} 
values may appear in any column and data row (except for the first data row). 
\item Empty values (i.e., missing values, \texttt{\small NULL}s, or 
\texttt{\small None}s) are represented by an empty string.  
\item A literal empty \textbf{string} is represented by two double-quotes: 
\texttt{""}, which is different from an \textbf{empty} (missing) value or 
\textbf{NaN}. 
\item The \textbf{string} values in \texttt{\small FILE} columns represent
a POSIX file path relative to the base directory (the \texttt{\small name.cdb} 
path) of the Cinema database. This is the location of the file data for that 
row.
\item The files can be of any type, where the format is indicated by MIME name 
extension.
\begin{enumerate}
\item Consumers of Spec D databases (readers, viewers, and analytics tools) 
are not required to display or read all formats. This is handled on a 
case-by-case basis for the use cases supported by the tool. Tools must 
provide appropriate error messages indicating when file formats are not 
supported. 
\item For clarity, we note here that browser-based viewers of Spec D are 
required to display the common web image file formats of JPEG, PNG, and GIF, 
and may ignore other file formats.
\end{enumerate}
\end{enumerate}
\end{enumerate}

\subsubsection{Examples}

\noindent
\textbf{Example 1} This type of \texttt{\small data.csv} file is 
self-contained with no \texttt{\small FILE} columns.

\begin{verbatim}
    timestep,time value,x,y,z,category
    1,0.1,1.0,1.1,1.2,one
    2,0.2,2.0,2.1,2.2,two
    3,0.3,3.0,3.1,3.2,three
    4,0.4,4.0,4.1,4.2,four
\end{verbatim}

\noindent
In this case, the Cinema database would contain the following files,
assuming that {\em database\_name} is the name of the database.

\begin{verbatim}
    database_name.cdb/
        data.csv
\end{verbatim}

\noindent
\textbf{Example 2} This \texttt{\small data.csv} example shows the 
\texttt{\small FILE} keyword on the last column and the corresponding
POSIX file paths.

\begin{verbatim}
    timestep,time value,x,y,z,category,FILE
    1,0.1,1.0,1.1,1.2,one,img/001.jpg
    2,0.2,2.0,2.1,2.2,two,img/002.jpg
    3,0.3,3.0,3.1,3.2,three,data/003.jpg
    4,0.4,4.0,4.1,4.2,four,data/004.png 
\end{verbatim}

\noindent
In this case, the Cinema database would contain the following files:

\begin{verbatim}
    database_name.cdb/
        data.csv
        img/
            001.jpg
            002.jpg
        data/
            003.jpg
            004.png
\end{verbatim}

\noindent
\textbf{Example 3} This \texttt{\small data.csv} example shows an example
of NULL, NaN, and empty string values in the last row.

\begin{verbatim}
    timestep,time value,x,y,z,category,FILE
    1,0.1,1.0,1.1,1.2,one,img/001.jpg
    2,0.2,2.0,2.1,2.2,two,img/002.jpg
    3,0.3,3.0,3.1,3.2,three,data/003.jpg
    4,0.4,4.0,4.1,4.2,four,data/004.png 
    ,0.5,,NaN,5.2,"",
\end{verbatim}

\noindent
In this case, the Cinema database would contain the same files as 
Example 2. \\

\noindent
\textbf{Example 4} This \texttt{\small data.csv} example shows multiple
\texttt{\small FILE}s on the last columns, and the corresponding POSIX file 
paths. Note that the file columns are named \texttt{\small FILE}
and \texttt{\small FILE plot}, in this example.

\begin{verbatim}
    timestep,time value,x,y,z,category,FILE,FILE plot
    1,0.1,1.0,1.1,1.2,one,img/001.jpg,plot/a.csv
    2,0.2,2.0,2.1,2.2,two,img/002.jpg,plot/b.tsv
    3,0.3,3.0,3.1,3.2,three,data/003.jpg,plot/c.txt
    4,0.4,4.0,4.1,4.2,four,data/004.png,plot/d.xls
    ,0.5,,NaN,5.2,"",,
\end{verbatim}

\noindent
In this case, the Cinema database would contain the following files:

\begin{verbatim}
    database_name.cdb/
        data.csv
        img/
            001.jpg
            002.jpg
        data/
            003.jpg
            004.png
        plot/
            a.csv
            b.tsv
            c.txt
            d.xls
\end{verbatim}

\noindent
\textbf{Example 5} This \texttt{\small data.csv} example shows string
quotation (rfc4180) to create a single text vector column from the
{\em x}, {\em y}, and {\em z} columns.

\begin{verbatim}
    timestep,time value,"x,y,z",category,FILE,FILE plot
    1,0.1,"1.0,1.1,1.2",one,img/001.jpg,plot/a.csv
    2,0.2,"2.0,2.1,2.2",two,img/002.jpg,plot/b.tsv
    3,0.3,"3.0,3.1,3.2",three,data/003.jpg,plot/c.txt
    4,0.4,"4.0,4.1,4.2",four,data/004.png,plot/d.xls
    ,0.5,,"",,
\end{verbatim}

\noindent
In this case, the Cinema database would contain the same files as Example 4.

\subsubsection{Data Types}

The second line (first data line) is used to determine the data type of the
columns.  We assume that built-in parsing in modern languages, such as in
Python and JavaScript, is available and can be used to test, parse, and convert 
values from the \texttt{\small csv} file.

In our previous {\bf Example 1}, the types of the columns are 
\textbf{integer}, four columns of \textbf{float}, and \textbf{string}. Parsing
precedence for determining the type of a column is: \textbf{integer},
\textbf{float}, \textbf{string}. That is, if a column in the first data row
(second line) parses as an \textbf{integer}, it is an \textbf{integer}
column, otherwise if it parses as a \textbf{float}, it is a \textbf{float}
column, otherwise it is a \textbf{string} column. This does mean that
if you have floating-point values that are integers, they need to be
written as floating point in the second line (first data row) to be able to 
type a column correctly, i.e., a 0 must be stored as 0.0.

For strings:
\begin{itemize}
\item \texttt{\small nan}, \texttt{\small NaN}, and \texttt{NAN} (or any
  combination of capitalization) are valid values, and should be handled
  by any readers and viewers according to the type of the column. If
  these are present in the first data row, the column is parsed as 
  \textbf{float} values. \texttt{\small NaN} is not a valid \textbf{integer}
  value for any row.
\item Whitespace is not consumed (stripped) before and after comma separators.
  This is part of the rfc4180 specification. Whitespace {\em cannot} appear
  before or after a comma that has a double-quoted value (an invalid rfc4180).
\item Commas and double-quotes are special characters that require 
  enclosing the value in double-quotes. A literal double-quote must be preceded 
  by another double-quote: e.g., \texttt{\small """"} is the value 
  \texttt{\small "} (please refer to rfc4180).
\item A literal empty string must be represented as \texttt{\small ""}, as an 
  empty string (no value between two commas, or before the first comma
  or after the last comma) is \texttt{\small NULL} (empty value).
\end{itemize}

