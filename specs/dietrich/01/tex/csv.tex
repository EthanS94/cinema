The main data file for this specification is a Comma Separated Value file that
has the following requirements.  The last data column(s) may reference one or
more external files. These referenced files can be of any file type or file
extension.  Note that new data can be added to or deleted from the
\texttt{\small csv} file or files can be added to or removed from the database
i.e., the \texttt{\small database\_name.csv} directory, as long as the
following requirements are met.} The specification for the Spec D v\version
\texttt{\small csv} follows (differing from v1.0 in points 5 and 9).

\begin{enumerate}
\item The file is UTF-8 encoded.
\item The file follows the specification rfc4180 \cite{rfc4180}, governing how
      fields are defined.
\item The first line of the file (header) is \textbf{required}.
\begin{enumerate}
    \item Header values are unique non-empty \textbf{strings} that are the
          labels for the columns.
    \item The last header values may be or start with \texttt{\small FILE}. No
          other non-\texttt{\small FILE} columns can occur after the first
          \texttt{\small FILE} columns (\texttt{\small FILE} columns are
          always last).
\end{enumerate}

\item The remaining lines in the file are data rows.

\begin{enumerate}
    \item There must be at least one data row.
    \item Each data line is populated with an equal number of values
          (columns) as the header (i.e., the same number of separating commas
          per row).
    \item All data values must be \textbf{floats} (floating-point values),
          \textbf{integers}, \textbf{strings}, or \textbf{empty} (missing)
          values.
    \item All columns are \textbf{required} to have the same value type for
          all data rows, either \textbf{float}, \textbf{integer}, or
          \textbf{string}.
    \item \textbf{Empty} values may appear in any column and data row, are represented by an empty string.
    \item A literal empty \textbf{string} is represented by two double-quotes:
          \texttt{""}, which is different from both an \textbf{empty} (missing)
          value or \textbf{NaN}.
    \item A data value representing a file path must be a string containing either
          a) a POSIX file path, or b) a URI.
    \begin{enumerate}
    \item Files can be of any type, as indicated by MIME name extension, e.g.,
          \textbf{.png}, \textbf{.vti}, \textbf{.txt}, etc.
    \end{enumerate}
% \begin{enumerate}
% \item Consumers of Spec D databases (readers, viewers, and analytics tools) are not required to display or read all file formats. This is handled on a case-by-case basis, for the specific use cases, supported by each implementation.  Implementations must provide appropriate error messages indicating when particular file formats are not supported.
% \begin{enumerate}
% \item For clarity, we note here that browser-based viewers of Spec D are required to display the common web image file formats of JPEG, PNG, and GIF, and may ignore other file formats.
% \end{enumerate}
\end{enumerate}
\end{enumerate}

\subsubsection{Data Types}

The first \textbf{non-empty} value is used to determine the data type of a
column.  We assume that built-in parsing in modern languages is available and
can be used to test, parse, and convert values from the \texttt{\small csv}
file.

In our previous examples {\bf 1} and {\bf 2}, the types of the columns are
\textbf{integer}, four columns of \textbf{float}, and \textbf{string}. For
the case of example {\bf 1}, determining the data types for columns 2 and 4
are deferred until the second row is parsed. All other subsequent values must
be the same data type or \textbf{empty}.

Parsing precedence for determining the type of a column is: \textbf{integer},
\textbf{float}, \textbf{string}. That is, if the first non-empty value in a
column parses as an \textbf{integer}, it is an \textbf{integer} column,
otherwise if it parses as a \textbf{float}, it is a \textbf{float} column,
otherwise it is a \textbf{string} column. \textit{In particular, this means
that floating-point values that are integers, must be written as floating point
value: i.e., they must contain a decimal point.} For example, a value of 0 in a
floating point column must be represented as 0.0.

Further explanation on particular values:
\begin{itemize}
\item \texttt{\small nan}, \texttt{\small NaN}, and \texttt{NAN} (or any
  combination of capitalization) are valid values, and should be handled by any
  implementation according to the type of the column, either \textbf{floats}
  or \textbf{strings}. If these are present in the \textit{first}
  \textbf{non-empty} data row, then data type of the column should be
  \textbf{float} values. \texttt{\small NaN} is not a valid \textbf{integer}
  value for \textit{any} row.
\item Whitespace is not and should not be consumed (stripped) before and after
  comma separators.  This is part of the rfc4180 specification. Whitespace {\em
  cannot} appear before or after a comma that has a double-quoted value, as
  that is an invalid rfc4180 file.
\item Commas and double-quotes are special characters that require
  enclosing the value in double-quotes. A literal double-quote must be preceded
  by another double-quote: e.g., \texttt{\small """} is the value
  \texttt{\small "}. (Please refer to rfc4180.)
\item A literal empty string must be represented as \texttt{\small ""}, as an
  empty string (no text between two commas, or before the first comma
  or after the last comma) is \texttt{\small NULL} (empty value). Whitespace
  between commas will be interpreted as the whitespace \textbf{not} as an
  empty string or empty value.
\end{itemize}

