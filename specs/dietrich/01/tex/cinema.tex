\label{sec:cinema}

Extreme scale scientific simulations are leading a charge to exascale computation, and data analytics runs the risk of being a bottleneck to scientific discovery. Due to power and I/O constraints, we expect in situ visualization and analysis will be a critical component of these workflows. 

Options for extreme scale data analysis are often presented as a stark contrast: write large files to disk for interactive, exploratory analysis, or perform in situ analysis to save detailed data about phenomena that a scientist knows about in advance. Cinema represents a novel framework for a third option – a highly interactive, data artifact-based approach that promotes exploration of simulation results, and is easily accessed through database specifications. This approach supports interactive exploration of a wide range of results, while still significantly reducing data movement and storage.

More information about the overall design of Cinema is available in the paper \textit{An Image-based Approach to Extreme Scale In Situ Visualization and Analysis} \cite{cinemaSC14}.

A Cinema Database supports the following three use cases. Taken together, these support a novel method for interactively exploring artifacts from extremely large datasets.

\begin{enumerate}
\item Searching/querying of meta-data and data artifacts. Samples can be searched purely on metadata, content, position, time, or a combination of all of these.
\item Interactive visualization of sets of data artifacts.
\item Playing interactive visualizations, allowing the user on/off control of elements in the visualization.
\end{enumerate}

\subsection{What is a Cinema Database?}
A Cinema database is a set of parameters mapped with a set of data artifacts. The artifacts can be images, grids, csv files - any type of data that can be written to disk.

A general design philosophy of Cinema is that applications reading and viewing a Cinema database can determine which data to operate on, and which operations to perform. This promotes a wide range of possible interactions with the data - not just the ones imagined by the creator of the database.

